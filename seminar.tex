% Master-Konferenzseminar, FH Suedwestfalen
%
% Vorlage ist xelatex-kompatibel
% Literatur in seminar.bib
% Build-Prozess:
%  xelatex seminar; bibtex seminar; xelatex seminar; xelatex seminar
%
%
\documentclass[twoside,a4paper]{IEEEtran}
\usepackage{fontspec}
\usepackage[ngerman]{babel}
\usepackage{cite}
\usepackage{xcolor}
\usepackage{listings}
\lstset{basicstyle=\footnotesize\ttfamily,breaklines=true}
\lstset{numberstyle=\tiny\color{blue},keywordstyle=\color{blue}}
\usepackage{blindtext}  % <-- kann weg!
\renewcommand{\thepage}{X-\arabic{page}}
\begin{document}
% paper title
% can use linebreaks \\ within to get better formatting as desired
\title{Das ist der Titel meiner \\ Seminararbeit%
    \thanks{%
    Dieser Beitrag entstand im 
    Rahmen des Master"=Konferenzseminars \emph{info@swf 2022}, das im
    Wintersemester 2021/22 vom Fachbereich Informatik und
    Naturwissenschaften der Fachhochschule Südwestfalen
    durchgeführt wurde.}}
% author names and affiliations
% use a multiple column layout for up to three different
% affiliations
\author{
  \IEEEauthorblockN{Vorname Nachname\\ }
  \IEEEauthorblockA{Fachhochschule Südwestfalen}
}

\maketitle


%%% Zusammenfassung
\begin{abstract}
  %
\blindtext
  %
\end{abstract}


%%% Text beginnt
\section{Einleitung}

This demo file is intended to serve as a ``starter file''
for IEEE conference papers produced under \LaTeX\ using
IEEEtran.cls version 1.7 and later \cite{tanenbaum2014modernos}.
Wir zitieren hier außerdem aus einem Werk zur Geschichte
von UNIX \cite{salus:0201547775:unix:history}.

\Blindtext[2]

Gerne verweisen wir an dieser Stelle auf eine Abbildung,
nämlich auf die Abbildung~\ref{bild1}. Das ist wirklich eine
schöne Abbildung.

%%% Bild
\begin{figure}[b]
  % \centering
  \includegraphics[width=\columnwidth]{bild}
  \caption{Ergebnisse der Simulation}
  \label{bild1}
\end{figure}


\section{Mehr Blindtext}

\Blindtext[3]

%%% Listing: inline
\begin{lstlisting}[language=C,label=listing1,caption={Das ist das erste Listing (inline.c).}]
int main () {
  printf ("Hello World!\n");
}
\end{lstlisting}

\Blindtext[1]


%%% Listing: als float
% Extra-Parameter: Zeilennummern, Listing-Zeilen 6mm einrücken
\begin{lstlisting}[language=Bash,float={t!},label=listing1,caption={Das ist das zweite Listing (float.sh).},numbers=left,xleftmargin=6mm]
alias ll='exa -bghHlBa'
alias ltr='ls -ltrG'
alias ..="cd .."
export PATH=$PATH:/Library/TeX/texbin:/Users/esser/bin:/opt/local/bin
\end{lstlisting}


%%% Tabelle
\begin{table}[t]
  \caption{Eine Tabelle mit Beispielinhalten}
  \label{table_example}
  \centering
  \begin{tabular}{|c||c|}
    \hline
    One & Two\\
    \hline
    Three & Four\\
    \hline
  \end{tabular}
\end{table}



%%% Literaturverzeichnis
\bibliographystyle{IEEEtran}
\bibliography{seminar}

%% oder manuell
%\begin{thebibliography}{1}
%
%\bibitem{IEEEhowto:kopka}
%H.~Kopka and P.~W. Daly, \emph{A Guide to \LaTeX}, 3rd~ed.\hskip 1em plus
%  0.5em minus 0.4em\relax Harlow, England: Addison-Wesley, 1999.
%
%\end{thebibliography}

\end{document}